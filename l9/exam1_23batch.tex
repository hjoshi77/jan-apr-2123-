\documentclass{exam}
\usepackage{amsmath}
\newcommand*\textfrac[2]{
  \frac{\text{#1}}{\text{#2}}
}
\begin{document}
\newcommand\ddfrac[2]{\frac{\displaystyle #1}{\displaystyle #2}}

\begin{center}
  \bfseries\large
  Department of Biotechnology\\
 Biomolecular simulations  BT2123\\
 Segment 1-3 exam \\
16 February 2024 \\ 

\begin{flushright} \textbf{Maximum Marks: 100} \end{flushright} 

 \bigskip
\fbox{\fbox{\parbox{5.5in}{\centering
The description or answer to a subjective question need not to be longer than 10 sentences}}}
\end{center}
\vspace{3mm}

\begin{questions} 
\question Compute the wavelength associated with \\ 
(a) an electron moving in 300 keV voltage (ignoring the relativistic effects) \\
(b) An Argon gas atom moving due to thermal energy at 300K. \\
Based on wavelengths, how do you will decide the when the quantum mechanical behavior is significant ? \\ 

 \question When did the field of molecular dynamics simulation started. Write down some milestones in the method of molecular dynamics with the name of four pioneering scientists. \\ 
 \question What are the major force-fields and simulation engines in MD simulations of biomolecules ?  \\
\question What do you understand from one MD cycle, write down the steps involved in one MD cycle.  \\
\question What is the functional form of the potential energy function (force-field) including the bonded and non-bonded terms used in molecular dynamics simulation, plot them schematically. \\ 
\question Given the form $\frac{A}{r^{12}} - \frac{B}{r^{6}}$ and derive $\epsilon_{ij}$ [$\frac{R_{min}}{r_{ij}^{12}}$  - 2 $\frac{R_{min}}{r_{ij}^6}$ ] for Lennard Jones potential.  Draw the Lennard Jones potential, show the  $\epsilon_{ij}$  and $R_{min}$ in the plot. \\ 
\question Write down the rotational and translational parameters of defining the double helical structure of DNA \\ 
\question What is PCR? Give three examples where it has shown promising applications. \\ 
\question What is the role of forcefields in MD simulations \\ 
\question What is Levinthal's paradox, give a brief history and techniques involved in protein crystallography.  \\ 

%V= 0.3 & \left[ $\left( \ddfrac{3} {r}\right)$^{12} -2 $\left( \ddfrac{3} {r}\right)$^{6}] \\  
\end{flushleft}
\end{enumerate}  
  
 
\end{questions}
\end{document}
